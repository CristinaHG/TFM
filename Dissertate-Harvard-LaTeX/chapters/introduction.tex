%!TEX root = ../dissertation.tex
\chapter{Introduction}
\label{introduction}

La predicción y clasificación supervisada de datos numéricos siempre ha sido un tema central de interés científico en el área de ciencia de datos y aprendizaje automático. También hay numerosas investigaciones publicadas en torno a el preprocesamiento de datos. Sin embargo, hasta hace unos años apenas se había prestado interés a la clasificación de datos ordinales, también conocida como regresión ordinal.

\section{El problema de la clasificación ordinal}
Los datos de naturalieza ordinal no son más que datos cuyas clases muestran un claro orden. Es decir, para cada instancia $x \in X \subseteq \Re^K $, con clase $y\in Y={C_1,C_2,...,C_n}$, se tiene una relación de orden sobre las clases tal que $C_1 \prec C_2 \prec ... \prec C_n$. Por ejemplo, datos recogidos de encuestas con diferentes grados de valoración, datos de resultados médicos considerando diferentes enfermedades, datos de psicología o, en general, datos obtenidos en cualquier área científica donde los humanos intervengan para la generación de datos. \newline
Por tanto, el problema de la clasificación de datos ordinales, o regresión ordinal, trata de clasificar datos ordinales considerando en los algoritmos el orden que presentan entre sí las clases de los datos, para obtener una clasificación más fiel a la naturaleza intrínseca de estos datos, y por tanto, construir un modelo más preciso. Pongamos como ejemplo datos obtenidos de una encuesta de valoración de un servicio que realizan un número de clientes, donde las posibles etiquetas son: \textit{$\left[malo, normal, bueno, muy\ bueno, excelente\right]$}. En este caso, las clases contienen información de orden, pues pertenecer a la clase \textit{malo} implica una peor valoración que pertenecer a la clase \textit{excelente}, y por tanto un algoritmo de clasificación diseñado para tratar con datos ordinales debería considerar diferentes costes según el tipo de errores de clasificación que se produjesen. Por ejemplo, debería de penalizar más el clasificar como \textit{malo} un servicio \textit{excelente} que clasificarlo como \textit{muy bueno}. Otro ejemplo similar de regresión ordinal sería la clasificación del riesgo de varias enfermedades en \textit{$\left[bajo, moderado, severo\right]$} en base a los síntomas presentados por los pacientes. En este caso, la relación de orden presente sería los niveles de gravedad de una enfermedad. \newline

\section{El problema de la clasificación monotónica}

\section{Aplicaciones}