\chapter{Manual de usuario de OCAPIS 1.0}
\section{Fselector}

\subsection{Tittle}
Feature Selection for Monotonic Classification.
\subsection{Use}
\begin{minted}
[frame=lines,
framesep=1mm,
fontsize=\footnotesize
]{r}
fselector(traindata, trainlabels, k, beta, nselected)
\end{minted}
\subsection{Arguments}

\hline
\vspace{5pt}
\begin{itemize}
	\item \textbf{traindata}: Training data of numeric type without labels.
	
	\item \textbf{trainlabels}: A vector of numeric tags for each instance of training data.
	
	\item \textbf{k}: positive constant for logit function. If large, fuzzy ordinal set is understood as slightly
	larger. If small, is understood as significantly larger.
	
	\item \textbf{beta}: Regulation param for relative importance of MI between features and decision.
	
	\item \textbf{nselected}: Number of features to select.

\vspace{5pt}
\hline
\end{itemize}

\subsection{Return}
nselected most important features.
\subsection{Description}
Selects the N most relevant features from ordinal and monotonic data, based on mRMR criterion.

\subsection{Examples}
\begin{minted}
[frame=lines,
framesep=1mm,
fontsize=\footnotesize
]{r}
dattrain<-read.table("train_balance-scale.0", sep=" ")
trainlabels<-dattrain[,ncol(dattrain)]
traindata=dattrain[,-ncol(dattrain)]
selected<-fselector(traindata,trainlabels,2,2,2)
\end{minted}

\section{Iselector}

\subsection{Tittle}
Instance Selection for Monotonic Classification
\subsection{Use}
\begin{minted}
[frame=lines,
framesep=1mm,
fontsize=\footnotesize
]{r}
iselector(traindata, trainlabels, candidates, collisions, kEdition)
\end{minted}
\subsection{Arguments}

\hline
\vspace{5pt}
\begin{itemize}
	\item \textbf{traindata}: Training data of numeric type without labels.
	
	\item \textbf{trainlabels}: A vector of numeric tags for each instance of training data.
	
	\item \textbf{candidates}:Rate of the best candidates to be selected.
	
	\item \textbf{collisions}: Minimal rate of collisions permitted to stop the removal process.
	
	\item \textbf{kEdition}: Maximum number of nearest neighborgs to consider.
	
	\vspace{5pt}
	\hline
\end{itemize}

\subsection{Return}
A reduced dataset with the selected instances and its labels.
\subsection{Description}
Selects the N most relevant instances from ordinal and monotonic dataset, using a three-steps algorithm.

\subsection{Examples}
\begin{minted}
[frame=lines,
framesep=1mm,
fontsize=\footnotesize
]{r}
dattrain<-read.table("train_balance-scale.0", sep=" ")
trainlabels<-dattrain[,ncol(dattrain)]
traindata=dattrain[,-ncol(dattrain)]
selected<-iselector(traindata,trainlabels,0.01,0.01,5)
\end{minted}

\section{kdlorpredict}

\subsection{Tittle}
Predicts KDLOR model outputs for test data.
\subsection{Use}
\begin{minted}
[frame=lines,
framesep=1mm,
fontsize=\footnotesize
]{r}
kdlorpredict(fittedmodel, trainData, testData)
\end{minted}
\subsection{Arguments}

\hline
\vspace{5pt}
\begin{itemize}
	\item \textbf{fittedmodel}: fitted model of class kdlorModel obtained with kdlortrain(...).
	
	\item \textbf{trainData}: Data used to previously train the model. Tags should not be provided in traindata.
	
	\item \textbf{testData}: Test data without labels.
	
	\vspace{5pt}
	\hline
\end{itemize}

\subsection{Return}
A list of two elements containing the predicted labels for each instances and the projected values.
\subsection{Description}
Predicts the test data labels using a Kernel Discriminant Learning for Ordinal Regression fitted model

\subsection{Examples}
\begin{minted}
[frame=lines,
framesep=1mm,
fontsize=\footnotesize
]{r}
testdata<-read.table("test_balance-scale.0", sep=" ")
testdata<-testdata[,-ncol(testdata)]
pred<-kdlorpredict(myfit,traindata,testdata)
\end{minted}


\section{kdlortrain}

\subsection{Tittle}
Trains a KDLOR model.
\subsection{Use}
\begin{minted}
[frame=lines,
framesep=1mm,
fontsize=\footnotesize
]{r}
kdlortrain(traindata, trainlabels, kernel, d, u, k)
\end{minted}
\subsection{Arguments}

\hline
\vspace{5pt}
\begin{itemize}
	\item \textbf{traindata}: Data to train kdlor model. Tags should not be provided in traindata.
	
	\item \textbf{trainlabels}: Class labels for training data. Must be numeric of type integer.
	
	\item \textbf{kernel}: Type of kernel to compute the Gram matrix. One of \textit{"rbf","gauss","gaussian","sigmoid", \\"linear","poly","polynomial"}.
	
	\item \textbf{u}: Numeric parameter for H matrix computation. Default is 0.01.
	
	\item \textbf{k}: Array of kernel Params. If kernel type is sigmoid, Array of two values should be provided.
	
	\item \textbf{c}: Numeric parameter for optimization method. Default is 10..
	
	\vspace{5pt}
	\hline
\end{itemize}

\subsection{Return}
An instance of kdlorModel class containing the fields: projectedTrain, predictedTrain, kerneltype, kernelParam, projection and thresholds, where: \newline
projectedTrain is the projected matrix for training data. \newline
predictedTrain are the predicted numeric labels for training data. \newline
kerneltype is the used kernel type for the fit. Should be used again for prediction.\newline
kernelParam is the numeric kernel param used for computing the kernel matrix.\newline
projection is the general projected matrix that should be used in predict.\newline
thresholds is an array of doubles representing the model thresholds to be used in prediction.\newline
Each of these fields can be accesed with "\@" (see section examples) below.
\subsection{Description}
Trains the Kernel Discriminant Learning for Ordinal Regression model with training data

\subsection{Examples}
\begin{minted}
[frame=lines,
framesep=1mm,
fontsize=\footnotesize
]{r}
# read train data
dattrain<-read.table("train_balance-scale.0", sep=" ")
traindata=dattrain[,-ncol(dattrain)]
trainlabels=dattrain[,ncol(dattrain)]
# fit the kdlor model
myfit<-kdlortrain(traindata,trainlabels,"rbf",10,0.001,1)
# acess kdlor model fields
myfit@predictedTrain
\end{minted}


\section{pomfit}

\subsection{Tittle}
Train a proportional odd model for ordinal regression.
\subsection{Use}
\begin{minted}
[frame=lines,
framesep=1mm,
fontsize=\footnotesize
]{r}
pomfit(train, trainLabels, linkfunction = "logistic")
\end{minted}
\subsection{Arguments}

\hline
\vspace{5pt}
\begin{itemize}
	\item \textbf{linkfunction}: link function to be used in the ordinal logistic regression fit. Possible functions are: logistic','probit','loglog','cloglog' or 'cauchit'.
	
	\item \textbf{x}: Training data of numeric type without labels.
	
	\item \textbf{y}: Tags for each instance of training data. Must be factors.
	
	\vspace{5pt}
	\hline
\end{itemize}

\subsection{Return}
the fitted model.
\subsection{Description}
train data must be the training data without labels. Labels should be provided in trainLabels.
\subsection{Examples}
\begin{minted}
[frame=lines,
framesep=1mm,
fontsize=\footnotesize
]{r}
dattrain<-read.table("train_balance-scale.0", sep=" ")
fit<-pomfit(dattrain[,-ncol(dattrain)],as.factor(dattrain[,ncol(dattrain)]),"logistic")
\end{minted}



\section{pompredict}

\subsection{Tittle}
Predict over the new data instances using the trained model.
\subsection{Use}
\begin{minted}
[frame=lines,
framesep=1mm,
fontsize=\footnotesize
]{r}
pompredict(model, test)
\end{minted}
\subsection{Arguments}

\hline
\vspace{5pt}
\begin{itemize}
	\item \textbf{model}: A trained POM model.
	\item \textbf{test}: Numeric test data without labels.
	\vspace{5pt}
	\hline
\end{itemize}

\subsection{Return}
A list containing at the first position the projected values per instance per class and at the second position the predicted label for the values.
\subsection{Description}
Predict over the new data instances using the trained model.
\subsection{Examples}
\begin{minted}
[frame=lines,
framesep=1mm,
fontsize=\footnotesize
]{r}
dattrain<-read.table("train_balance-scale.0", sep=" ")
fit<-pomfit(dattrain[,-ncol(dattrain)],as.factor(dattrain[,ncol(dattrain)]),"logistic")
dattest<-read.table("test_balance-scale.0", sep=" ")
predictions<-pompredict(fit,dattest[,-ncol(dattest)])
projections<-predictions[[1]]
predictedLabels<-predictions[[2]]
\end{minted}



\section{svmofit}

\subsection{Tittle}
Train n-1 SVM models for ordinal data with given parameters.
\subsection{Use}
\begin{minted}
[frame=lines,
framesep=1mm,
fontsize=\footnotesize
]{r}
svmofit(train, trainLabels, weights = TRUE, cost, gamma)
\end{minted}
\subsection{Arguments}

\hline
\vspace{5pt}
\begin{itemize}
	\item \textbf{weights}: A boolean indicating whether weights per instance are used.
	\item \textbf{cost}: numeric value indicating the cost parameter to train the SVM.
	\item \textbf{gamma}:numeric value indicating the gamma parameter to train the SVM.
	\item \textbf{train}: Training data of numeric type without labels.
	\item \textbf{trainLabels}: A vector of numeric tags for each instance of training data.
	\vspace{5pt}
	\hline
\end{itemize}

\subsection{Return}
A matrix of 1xn svm trained with weights models.
\subsection{Description}
Train data must be the data without labels. Labels should be provided in trainLabels
\subsection{Examples}
\begin{minted}
[frame=lines,
framesep=1mm,
fontsize=\footnotesize
]{r}
dattrain<-read.table("train_balance-scale.0", sep=" ")
modelstrain<-svmofit(dattrain[,-ncol(dattrain)],dattrain[,ncol(dattrain)],TRUE,1,1)
\end{minted}


\section{svmopredict}

\subsection{Tittle}
Predict over the new data instances using the trained models.
\subsection{Use}
\begin{minted}
[frame=lines,
framesep=1mm,
fontsize=\footnotesize
]{r}
svmopredict(models, test)
\end{minted}
\subsection{Arguments}

\hline
\vspace{5pt}
\begin{itemize}
	\item \textbf{models}: A matrix of 1xN trained SVM models. Where N denotes the number of classes of the problem minus one.
	\item \textbf{test}: Numeric test data without labels.
	\vspace{5pt}
	\hline
\end{itemize}

\subsection{Return}
A list containing the projected values per instance per class and the predicted values (the maximum probability for each data instance).
\subsection{Description}
Predict over the new data instances using the trained models.
\subsection{Examples}
\begin{minted}
[frame=lines,
framesep=1mm,
fontsize=\footnotesize
]{r}
dattrain<-read.table("train_balance-scale.0", sep=" ")
modelstrain<-svmofit(dattrain[,-ncol(dattrain)],dattrain[,ncol(dattrain)],TRUE,1,1)
dattest<-read.table("test_balance-scale.0", sep=" ")
predictions<-svmopredict(modelstrain,dattest[,-ncol(dattest)])
\end{minted}



\section{wknnor}

\subsection{Tittle}
K-nearest neighbors for ordinal and monotonic data
\subsection{Use}
\begin{minted}
[frame=lines,
framesep=1mm,
fontsize=\footnotesize
]{r}
wknnor(traindata, trainlabels, testdata, k, q, kerneltype, monotonicity)
\end{minted}
\subsection{Arguments}

\hline
\vspace{5pt}
\begin{itemize}
	\item \textbf{traindata}: Training data of numeric type without labels.
	\item \textbf{trainlabels}: A vector of numeric tags for each instance of training data.
	\item \textbf{testdata}: Test data of numeric type.
	\item \textbf{q}: Minkowski distance param. Use q=1 for Manhattan distance and q=2  for Euclidean distance.
	\item \textbf{kerneltype}: Kernel used to compute neighbors weights. Avaiable kernels are: rectangular, triangular,epanechnikov,
		biweight, triweight, cosine, gauss and inversion.
	\item \textbf{monotonicity}: Boolean param specifying whether data is monotone or not.
	\vspace{5pt}
	\hline
\end{itemize}

\subsection{Return}
Predicted labels for test data.
\subsection{Description}
Predicts labels for test data with ordinal nature or monotonic constraints using an adaptation
of wknn.
\subsection{Examples}
\begin{minted}
[frame=lines,
framesep=1mm,
fontsize=\footnotesize
]{r}
dattrain<-read.table("train_balance-scale.0", sep=" ")
traindata=dattrain[,-ncol(dattrain)]
trainlabels=dattrain[,ncol(dattrain)]
testdata<-read.table("test_balance-scale.0", sep=" ")
testdata<-testdata[,-ncol(testdata)]
testlabels<-wknnor(traindata,trainlabels,testdata,5,2,"rectangular",FALSE)
\end{minted}