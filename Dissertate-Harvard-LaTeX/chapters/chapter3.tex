%!TEX root = ../dissertation.tex

\chapter{Descripción del Software desarrollado}

Ésta es la primera versión de OCAPIS, una librería software para preprocesamiento y clasificación de datos ordinales y monotónicos desarrollada para R pero implementada en Scala mayoritariamente, para mejorar su rendimiento. Para comunicar R con Scala se hace uso del paquete \textbf{rScala} \cite{dahlintegration}.

\section{Instalación}
Como dependencias a nivel de sistema, el paquete requiere que el usuario tenga instaladas alguna versión de Python y alguna versión de Scala en su sistema. Si este requisito no se cumple, el paquete no funcionará. El requerimiento de \textit{Scala} es debido a que el serializer necesitará que esté instalado en el sistema para poder hacer las comunicaciones. 
\subsection{Instalación de Scala}
Scala puede instalarse desde los repositorios, en caso de usar linux:
\vspace{3pt}
\hline
\vspace{7pt}
 \$ sudo apt-get install scala
 \vspace{7pt}
 \hline
 \vspace{3pt}
En caso de Ubuntu, sustituyendo el comando por \textit{emerge} si la distribución usada es gentoo o el correspondiente comando de instalación. Scala también puede instalarse desde su página web: \url{https://www.scala-lang.org/download/install.html}. \newline

\subsection{Instalación de Python}
Python se instala de forma similar, vía el repositorio o la página oficial : \url{https://www.python.org/downloads/}. Python es requerido porque la SVM ordinal implementada usa una librería C con wrapper para python para aplicar pesos por instancias, de nombre \textbf{libsvm-weights}.

\subsection{Instalación de Libsvm-weights}
Esta librería es muy potente, pero no la podemos encontrar en ningún repositorio. Es por tanto que debemos descargarla desde github: \url{https://github.com/claesenm/EnsembleSVM/tree/master/libsvm-weights-3.17} y compilarla para windows o linux, según nuestro sistema, usando el makefile correspondiente (seguir las instrucciones del README de la libería). \newline
Por último debemos colocar la libraría compilada en el directorio de librerías de usuario: \textcolor{SchoolColor}{/usr/local/lib} o \textcolor{SchoolColor}{/usr/lib} en Linux, o su homólogo en windows. 

\subsection{Instalación de OCAPIS}
Actualmente OCAPIS puede descargarse desde Github, usando el paquete \textit{devtools} desde R:
\vspace{3pt}
\hline
\vspace{7pt}
\$ devtools::install\_github("CristinaHG/OCAPIS")
\vspace{7pt}
\hline
\vspace{3pt}\newline
Aunque pronto podrá descargarse desde CRAN. El resto de dependencias R del paquete se instalarán automáticamente.
\section{Uso}
Todos los algoritmos de clasificación de OCAPIS tienen un método asociado para entrenar con los datos de entrenamiento y otro para clasificar las instancias de test con el modelo aprendido, siguiendo un enfoque típico en los paquetes de R. Por ejemplo, KDLOR incluye un método \textit{kdlortrain} y otro \textit{kdlorpredict}, SVMOP incluye otros dos métodos \textit{svmofit} y \textit{svmopredict}, etc, cuyo uso se detalla en el manual de usuario. \newline
Por ejemplo, así sería el uso de SVMOP con el dataset ordinal \textcolor{SchoolColor}{balance}:
\begin{minted}
[frame=lines,
framesep=1mm,
fontsize=\footnotesize
]{r}
  library("OCAPIS")
   # lectura de datos de train
   dattrain<-read.table("train_balance-scale.0", sep=" ")
   # entrenamiento del modelo
   modelstrain<-svmofit(dattrain[,-ncol(dattrain)],dattrain[,ncol(dattrain)],TRUE,1,1)
   # lectura de datos de test
   dattest<-read.table("test_balance-scale.0", sep=" ")
   # cálculo de predicciones
   predictions<-svmopredict(modelstrain,dattest[,-ncol(dattest)])
\end{minted}

La documentación completa para estos métodos puede consultarse con
\begin{minted}
[frame=lines,
framesep=1mm,
fontsize=\footnotesize
]{r}
?OCAPIS::svmofit 
?OCAPIS::svmopredict
\end{minted}