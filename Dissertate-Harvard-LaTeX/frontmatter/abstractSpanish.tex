%!TEX root = ../dissertation.tex
% the abstract

El presente trabajo de fin de máster desarrolla un paquete software de múltiples algoritmos para la eficiente clasificación y preprocesamiento de datos con restricciones intrínsecas específicas debido a su propia naturaleza, como son los datos ordinales y monotónicos. \newline
En el primer capítulo se introduce el problema de trabajar con datos ordinales y monotónicos, su naturaleza y aplicaciones. El segundo capítulo ofrece una explicación teórica de  los cuatro algoritmos supervisados de clasificación desarrollados (SVMOP, KDLOR, POM and WKNN) y los dos de preprocesamiento (un  selector de instancias y un selector de características). El tercer capítulo instroduce el lenguaje de programación Scala empleado en el desarrollo de este paquete. El capítulo cuarto, incluye una descripción del desarrollo e instalación del paquete desarrollado para R, de nombre OCAPIS. El capítulo quinto contiene el manual de usuario que será distribuido junto con el paquete en CRAN. Finalmente, en el capítulo sexto se recogen resultados, conclusiones y futuras ideas sobre OCAPIS.
\newline \newline
\text{KEYWORDS: Aprendizaje automático, Clasificación ordinal, Clasificación monotónica, }
\newline \text{Preprocesamiento ordinal, Preprocesamiento Monotónico. }