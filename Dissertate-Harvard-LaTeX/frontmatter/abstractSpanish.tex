%!TEX root = ../dissertation.tex
% the abstract

El presente trabajo de fin de máster desarrolla un paquete software para R de múltiples algoritmos para la eficiente clasificación y preprocesamiento de datos con restricciones intrínsecas específicas debido a su propia naturaleza, como son los datos ordinales y monotónicos. \newline
En concreto se implementan en Scala cuatro técnicas de clasificación ordinal: Support Vector Machine for Ordinal Regression, Proportional Odd models, Kernel Discriminant Learning for Ordinal Regression, y Weighted-nearest-neighborgs, y dos algoritmos de preprocesamiento: un selector de características basado en teoría de la información que usa el criterio de Minimum Redundancy Maximum Relevance  junto con Rank Mutual Information, y un selector de instancias trifásico que primero selecciona características, porteriormente elimina colisiones y por último detecta las instancias más relevantes según medidas de delimitación e interés.
\newline \newline
\text{KEYWORDS: Aprendizaje automático, Clasificación ordinal, Clasificación monotónica, }
\newline \text{Preprocesamiento ordinal, Preprocesamiento Monotónico. }